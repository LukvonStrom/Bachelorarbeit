\begin{abstract}
\thispagestyle{kapitelkopfzeile}
\begin{center}
    \Large
    \textbf{Abstract}
\end{center}
Due to the rising interest in timeseries analysis coupled with the sharp incline of \acf{IoT} focused projects, SPIRIT/21 GmbH aims to maximize the value from all the data gathered. To achieve this, reference architectures are needed to handle the construction of scalable data analysis architectures in the cloud.
The goal of this bachelorthesis is, to model reference architectures for performing timeseries analytics in the \acf{AWS} Cloud, where most of the business activities of SPIRIT/21 GmbH in terms of cloud computing currently take place.
There are two modes of analysis, that are compared in this thesis: (near) realtime and batch analysis. The thesis thrives to achieve the construction of one reference architecture for each mode and compare them afterwards.
To construct the reference architectures, a set of offerings from \ac{AWS} is to be examined and compared against each other.
To ensure maximum utility, feedback and input is collected in various forms throughout the construction process from several stakeholders. The proceedings of this thesis including the drawings and the source code are opensourced after completion.

\end{abstract}

