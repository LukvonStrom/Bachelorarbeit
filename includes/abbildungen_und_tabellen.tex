\chapter{Beispiele für Abbildungen und Tabellen}\label{chapter:abbildungenTabellen}

Hier finden Sie Beispiele für Abbildungen, Tabellen, Formelsatz und  Source Code.

\section{Abbildungen}
In diesem Abschnitt gibt die Abbildungen~\ref{abb:Logo2cmHoch} und~\ref{abb:Logo2cmBreit}, die beide das Logo der DHBW zeigen.

\begin{figure}[htb]
\centering
\includegraphics[height=2cm]{graphics/dhbw.png}
\caption[DHBW-Logo 2cm hoch]{DHBW-Logo 2cm hoch.\footnotemark}
\label{abb:Logo2cmHoch}
\end{figure}
\footnotetext{Mit Änderungen entnommen aus: \cite{OhneAutorenOhneJahr}}

\lstset{language=TeX, % hervorzuhebende Keywords definieren
  morekeywords={footnotetext,footnotemark,footcite,caption}
}

\emph{Spezialfall:} Sofern \emph{innerhalb} der Bezeichnung einer Abbildung eine Fußnote angegeben oder eine Quelle referenziert werden soll, geschieht dies nicht per \lstinline|\footnote| oder \lstinline
|\footcite|. Vielmehr sind die Befehle \lstinline|\footnotemark| und \lstinline|\footnotetext| zu verwenden und außerdem das optionale Argument für \lstinline|\caption| anzugeben (vgl.\ Source Code).

\begin{figure}[htb]
\centering
\includegraphics[width=2cm]{graphics/dhbw.png}
\caption[DHBW-Logo 2cm breit.]{DHBW-Logo 2cm breit. (Quelle: DHBW\footnotemark)}
\label{abb:Logo2cmBreit}
\end{figure}
\footnotetext{\url{www.dhbw.de}}



\section{Tabellen}

In diesem Abschnitt gibt es zwei Beispiel-Tabellen, nämlich auf Seite~\pageref{tab:BeispielTabelleKlein} und auf Seite~\pageref{tab:BeispielTabelleGroesser}.

\begin{table}[htb]
\centering
\begin{tabular}{lcr}
links & Mitte & rechts \\
\hline
Muster & Muster & Muster \\
\end{tabular}
\caption{Kleine Beispiel-Tabelle.}
\label{tab:BeispielTabelleKlein}
\end{table}

\begin{table}[htb]
\centering
\begin{tabular}{|l|l|c|l|r||l}
    \textbf{Spalte 1} & \textbf{Spalte 2} & \textbf{Spalte 3} & \textbf{Spalte 4} & \textbf{Spalte 5} & \textbf{Spalte 6} \\
    \hline
    a        & b          & c                & d        & e        & f        \\
    Test     & Test, Test & Test, Test, Test & ~        & ~        & ~        \\
    1        & 2          & 3                & 4        & 5        & 6        \\
\end{tabular}
\caption{Größere Beispiel-Tabelle.}
\label{tab:BeispielTabelleGroesser}
\end{table}

\section{Etwas Mathematik}

Eine abgesetzte Formel:
\[
  \int_a^b x^2 \: \mathrm{d} x = \frac{1}{3} (b^3 - a^3)
\]

Es ist $a^2+b^2 = c^2$ eine Formel im Text.

\section{Source Code}

Source Code-Blöcke können auf folgende Arten eingefügt werden:

\lstset{language=Java}

Direkt im \LaTeX-Source Code:
\begin{lstlisting}
if(1 > 0) {
  System.out.println("OK"); 
} else {
  System.out.println("merkwuerdig");
}
\end{lstlisting}

oder eingefügt aus einer externen Datei.
\lstinputlisting{includes/HelloWorld.java}