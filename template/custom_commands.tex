%% Based on the Diagram by romeovs - https://tex.stackexchange.com/a/10064
\def\n{3}   
\def\N{5}   
\newcommand{\spider}[3]{
    \begin{tikzpicture}[]
        \foreach \x in{0,1,...,\n}{%
            \draw[->] (0,0)--(360/\n*\x:\N+0.5);
            \foreach \y in{0,1,...,\N}{
                \draw[thin,lightgray](360/\n*\x:\y)--(360/\n*\x+360/\n:\y);
                \draw[fill] (360/\n*\x:\y) circle(1pt);
                \ifnum\x=1
                    \draw(360/\n*\x:\y)node[above right]{\y};
                \fi
                }

        }

        \draw(360/\n:\N+0.5)node[above]{Dekompositionstiefe};   
        \draw(2*360/\n:\N+0.5)node[below]{Allgemeingültigkeit};   
        \draw(3*360/\n:\N+0.5)node[above=0.5cm]{Anwendbarkeit};       
        \draw[ultra thick,draw=blue](360/\n:#1)--(360/\n*2:#2)--(360/\n*3:#3)--cycle;
    \end{tikzpicture}
}

\newcommand{\spideroverview}[9]{
    \begin{tikzpicture}[]
        \foreach \x in{0,1,...,\n}{%
            \draw[->] (0,0)--(360/\n*\x:\N+0.5);
            \foreach \y in{0,1,...,\N}{
                \draw[thin,lightgray](360/\n*\x:\y)--(360/\n*\x+360/\n:\y);
                \draw[fill] (360/\n*\x:\y) circle(1pt);
                \ifnum\x=1
                    \draw(360/\n*\x:\y)node[above right]{\y};
                \fi
                }

        }

        \draw(360/\n:\N+0.5)node[above]{Dekompositionstiefe};
        \draw(2*360/\n:\N+0.5)node[below]{Allgemeingültigkeit};   
        \draw(3*360/\n:\N+0.5)node[above=0.5cm]{Anwendbarkeit};        
        
        \draw[ultra thick,draw=red](360/\n:#1)--(360/\n*2:#2)--(360/\n*3:#3)--cycle;
        \draw[ultra thick,draw=blue](360/\n:#4)--(360/\n*2:#5)--(360/\n*3:#6)--cycle;
        \draw[ultra thick,draw=cyan](360/\n:#7)--(360/\n*2:#8)--(360/\n*3:#9)--cycle;
        \begin{axis}[%
        hide axis,
        xmin=10,
        xmax=50,
        ymin=0,
        ymax=0.4,
        legend style={draw=white!15!black,legend cell align=left}
        ]
    \addlegendimage{white,fill=red,area legend}
    \addlegendentry{Philip A.};
    \addlegendimage{white,fill=blue,area legend}
    \addlegendentry{Ralph B.};
    \addlegendimage{white,fill=cyan,area legend}
    \addlegendentry{Peter E.};
    \end{axis}
    \end{tikzpicture}
}

\newcommand{\AWS}{\ac{AWS}}
\newcommand{\AWSIOT}{\ac{AWS}~\ac{IoT}}
\newcommand{\nzitat}{${}^{,}$}
\newcommand{\coo}{\ensuremath{\mathrm{CO_2}}}
\newcommand{\TodoW}[1]{
\PackageWarning{BA-Todos}{Todo-#1} 
\Todo{#1}
}

\newcommand{\anhangref}[1]{Anhang~\ref{#1}}

% \renewcommand{\clearpage}{}
\renewcommand*{\lstlistingname}{Codebeispiel}

%from https://groups.google.com/g/comp.text.tex/c/dOHk1iL9VKs/m/z7Yl8Kd0NUoJ - \hlinewd
\makeatletter
\def\hlinewd#1{%
\noalign{\ifnum0=`}\fi\hrule \@height #1 \futurelet
\reserved@a\@xhline}
\makeatother


\providecommand\phantomsection{}

% Siehe https://golatex.de/viewtopic.php?t=17077
\makeatletter
\newcommand{\textlabel}[2]{
  \protected@edef\@currentlabel{#2}
  \phantomsection
  #1\label{#2}
}
\makeatother

\newcommand{\vp}[1]{\textlabel{\protect\textbf{Variationspunkt~#1}}{Variationspunkt~#1}}

\newcommand{\vpref}[1]{\ref{Variationspunkt~#1}}

\newcommand{\englishquote}[1]{$\lbrack$\textit{#1}$\rbrack$}

\newcommand{\pandasmethod}[1]{\mintinline[breaklines]{python}{pandas.#1()}}

\newcommand{\miniabschnitt}[1]{\textbf{#1}\hfill\break\noindent}

\newcommand{\ftanhang}[1]{
\anhangteil{#1}\label{anhang:vergleich-#1} 
\miniabschnitt{Features von #1}\label{anhang:vergleich-features-#1}}

\newcommand{\dlanhang}[1]{\miniabschnitt{Dienstleistungsumfang von #1}\label{anhang:vergleich-umfang-#1}}

\newcommand{\weitereevaluationen}[1]{
\miniabschnitt{Weitere Evaluationen}
Die vom Dienst angebotenen Features sowie der Dienstleistungsumfang werden in \ref{anhang:vergleich-#1} diskutiert.
}




