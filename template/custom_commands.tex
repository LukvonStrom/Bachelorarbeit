%% Based on the Diagram by romeovs - https://tex.stackexchange.com/a/10064
\def\n{3}   %define how much axes you want
\def\N{5}   %define the number of node on each axis
\newcommand{\spider}[3]{
    \begin{tikzpicture}[]
        \foreach \x in{0,1,...,\n}{%
            \draw[->] (0,0)--(360/\n*\x:\N+0.5);
            \foreach \y in{0,1,...,\N}{
                \draw[thin,lightgray](360/\n*\x:\y)--(360/\n*\x+360/\n:\y);
                \draw[fill] (360/\n*\x:\y) circle(1pt);
                \ifnum\x=1
                    
                %     % \ifnum\y>0
                %         \draw(360/\n*\x:\y)node[below]{\y};
                %     % \else 
                %         % \draw(360/\n*\x:\y)node[above right]{\y};
                %     % \fi
                % \else
                    \draw(360/\n*\x:\y)node[above right]{\y};
                \fi
                }

        }

        \draw(360/\n:\N+0.5)node[above]{Dekompositionstiefe};     %adjust the labels (also add or delete exces axes)
        \draw(2*360/\n:\N+0.5)node[below]{Allgemeingültigkeit};   %eg. if you have 6 axes, delete the last 2
        \draw(3*360/\n:\N+0.5)node[above=0.5cm]{Anwendbarkeit};        %or if you have 9 axes add one
        \draw[ultra thick,draw=blue](360/\n:#1)--(360/\n*2:#2)--(360/\n*3:#3)--cycle;
    \end{tikzpicture}
}

\newcommand{\spideroverview}[9]{
    \begin{tikzpicture}[]
        \foreach \x in{0,1,...,\n}{%
            \draw[->] (0,0)--(360/\n*\x:\N+0.5);
            \foreach \y in{0,1,...,\N}{
                \draw[thin,lightgray](360/\n*\x:\y)--(360/\n*\x+360/\n:\y);
                \draw[fill] (360/\n*\x:\y) circle(1pt);
                \ifnum\x=1
                    \draw(360/\n*\x:\y)node[above right]{\y};
                \fi
                }

        }

        \draw(360/\n:\N+0.5)node[above]{Dekompositionstiefe};
        \draw(2*360/\n:\N+0.5)node[below]{Allgemeingültigkeit};   
        \draw(3*360/\n:\N+0.5)node[above=0.5cm]{Anwendbarkeit};        
        
        \draw[ultra thick,draw=red](360/\n:#1)--(360/\n*2:#2)--(360/\n*3:#3)--cycle;
        \draw[ultra thick,draw=blue](360/\n:#4)--(360/\n*2:#5)--(360/\n*3:#6)--cycle;
        \draw[ultra thick,draw=cyan](360/\n:#7)--(360/\n*2:#8)--(360/\n*3:#9)--cycle;
        % \node[draw=black,thick,rounded corners=2pt,below=5.5cm] {%
        % \begin{tabular}{p{5.05mm}|p{5.05mm}c|p{5.05mm}p{14.5mm}}
        %  \raisebox{2pt}{\tikz{\draw[ultra thick,draw=red] (0,0) -- (5mm,0);}} & Philip A. &
        %  \raisebox{2pt}{\tikz{\draw[ultra thick,draw=blue] (0,0) -- (5mm,0);}} &Ralph B. &
        %  \raisebox{2pt}{\tikz{\draw[ultra thick,draw=cyan] (0,0) -- (5mm,0);}}&Peter E.\\
        % \end{tabular}};
        \begin{axis}[%
        hide axis,
        xmin=10,
        xmax=50,
        ymin=0,
        ymax=0.4,
        legend style={draw=white!15!black,legend cell align=left}
        ]
    \addlegendimage{white,fill=red,area legend}
    \addlegendentry{Philip A.};
    \addlegendimage{white,fill=blue,area legend}
    \addlegendentry{Ralph B.};
    \addlegendimage{white,fill=cyan,area legend}
    \addlegendentry{Peter E.};
    \end{axis}
    \end{tikzpicture}
}

\newcommand{\AWS}{\ac{AWS}}
\newcommand{\AWSIOT}{\ac{AWS}~\ac{IoT} }
\newcommand{\nzitat}{${}^{,}$}
\newcommand{\coo}{\ensuremath{\mathrm{CO_2}}}

\newcommand{\interview}[6]{
% name
% position
% acro
% thema
% datum
\anhang{Experteninterview #1}\label{chap:interview-#6-#5}
\begin{table}[H]
\begin{tabularx}{\textwidth}{|l|X|}
\hline
    Datum                  & #5 \\ \hline
    Thema                  & #4 \\ \hline
    \begin{tabular}[c]{@{}l@{}}Teilnehmende,\\ Position\end{tabular} & \begin{tabular}[c]{@{}l@{}}Lukas Fruntke, Verfasser\\ #1, #2 - \acf{#3}\end{tabular}\\ \hline
\end{tabularx}
% \caption{Interviewübersicht #1}
% \label{tab:interviewuebersicht-#6-#5}
\end{table}
}

\newcommand{\produktbewertung}[2]{
\setsepchar{,}
\readlist\arg{#2}
\ifthenelse{\equal{#1}{Produkt}}
    {}
    {\subsubsection{#1}}

\begin{table}[H]
\centering
\begin{tabular}{|l|l|l|}
\hline
\rowcolor[HTML]{ECF4FF} 
Robustness and   fault tolerance & Scalability & möglichst serverless \\ \hline
\textbf{\arg[1]}/11 & \textbf{\arg[2]}/10 & \textbf{\arg[3]}/9 \\ \hline
\rowcolor[HTML]{ECF4FF} 
Minimal maintenance & Debuggability/transparente Fehler & Extensibility \\ \hline
\textbf{\arg[4]}/8 & \textbf{\arg[5]}/7 & \textbf{\arg[6]}/6 \\ \hline
\rowcolor[HTML]{ECF4FF} 
Low latency reads and updates & Ad hoc queries & Generalization \\ \hline
\textbf{\arg[7]}/5 & \textbf{\arg[8]}/4 & \textbf{\arg[9]}/3 \\ \hline
\rowcolor[HTML]{ECF4FF} 
Integration mit AWS & Übertragbarkeit (ISO 9126) & Summe \\ \hline
\textbf{\arg[10]}/2 & \textbf{\arg[11]}/1 & \cellcolor[HTML]{DAE8FC}\textbf{\arg[12]}/66 \\ \hline
\end{tabular}
\caption{Bewertungsmatrix #1}
\label{tab:bewertungsmatrix-#1}
\end{table}
}

% \renewcommand{\clearpage}{}
\renewcommand*{\lstlistingname}{Codebeispiel}