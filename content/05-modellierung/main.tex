\chapter{Modellierung}
\section{Anforderungserhebung}
Wie in \autoref{theorie:referenzmodellierung} beschrieben, müssen Referenzmodelle einen subjektiven Empfehlungscharakter besitzen, damit sie akzeptiert und wiederverwendet werden. Dafür muss ein Abgleich mit den Anforderungen der Nutzenden geschehen. Um dies zu erreichen, wurden im Anhang transkribierte Interviews (Vgl. Anhang~\ref{chap:interview-philipp-22.03.2021},\ref{chap:interview-ralph-24.03.2021},\ref{chap:interview-peter-24.03.2021}) durchgeführt. Daraus ergibt sich das in \autoref{abb:TopLevelEchtzeitRA} gezeigte Diagramm, welches die Anforderungen der individuellen Stakeholder an Dekompositiontiefe, Anwendbarkeit und Allgemeingültigkeit darstellt.


\Todo{Anforderungstabelle}

\begin{figure}[H]
\centering
\spideroverview
%{P. Arnold}
{5}{3}{3}
%{R. Briegel}
{3}{3}{1}
%{P. Erbacher}
{2}{4}{5}
\caption{Ergebnisse der Interviews}
\label{abb:DimensionenUebersicht}
\end{figure}
\section{Echtzeitverarbeitung}
\textbf{Noch nicht final, soll zeigen wie es mal werden soll}
\begin{figure}[H]
\centering
\includegraphics[width=\textwidth]{graphics/Echtzeit-RA-Overview.pdf}
\caption{Top Level View Referenzarchitektur}
\label{abb:TopLevelEchtzeitRA}
\end{figure}

\section{Datenbankseitige Verarbeitung}



\begin{figure}[H]
\centering
\includegraphics[width=\textwidth]{graphics/DB-RA-Overview.pdf}
\caption{Top Level View Referenzarchitektur}
\label{abb:TopLevelDBRA}
\end{figure}

\section{Einsatzszenarien der Referenzmodelle}