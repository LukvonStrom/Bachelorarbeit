\section{Bestehende Anwendungsfälle}
In diesem Unterkapitel werden die bestehenden Anwedungsfälle erläutert, welche innerhalb der SPIRIT/21 Daten zur Analyse liefern können.
%\subsection{Luftqualitätssensoren}

\subsection{Raumklimamonitoring}
Innerhalb des Hauptsitzes der SPIRIT/21 in Böblingen wurden im Rahmen der Covid-19 Bekämpfung Raumklimasensoren in allen Besprechungsräumen installiert, welche kontinuierlich die \coo{} Konzentration der Raumluft messen. Die \coo{}-Konzentration dient dabei laut aktueller Studienlage als Messhilfe für die Konzentration von möglichen Virenaerosolen im Raum und zur Lüftungsindikation, wenn ein kritischer Wert überschritten wird.\footcite[Vgl.][]{Hartmann.2020}\nzitat\footcite[Vgl.][]{Peng.2020} Zur Messung wurden die \enquote{ERS-\coo{}} Sensoren von Elsys in den Besprechungsräumen installiert. Zum Zeitpunkt der Fertistellung dieser Arbeit existieren 8 Sensoren in Besprechungsräumen in Böblingen. Die Daten werden mittels dem Funkstandard \ac{LoRaWAN} an ein Gateway gesendet, welches die Daten mittels \ac{MQTT} an ein Backend, wie beispielsweise die Open Source Low-Code Plattform Node-Red übermittelt. 

\begin{table}[H]
\centering
\begin{tabular}{|l|l|l|}
\hline
Attribut    & Datentyp & Einheit           \\ \hline
deviceName  & String   & -                 \\ \hline
temperature & Double   & \textdegree{}C     \\ \hline
humidity    & Integer  & \%                \\ \hline
light       & Integer  & Lux               \\ \hline
motion      & Integer  & Anzahl Bewegungen \\ \hline
$co_2$        & Integer  & ppm               \\ \hline
$v_{dd}$ (Spannung)         & Integer  & mV                \\ \hline
\end{tabular}
\caption[Datenschema Elsys ERS \coo{}~Sensor]{Datenschema Elsys ERS \coo{}~Sensor.\footnotemark}
\label{tab:data-schema-elsys}
\end{table}
\footnotetext{Mit Änderungen entnommen aus: \cite{ELSYS.2019}}
Nach Übermittlung in das Backend haben die Daten das in \autoref{tab:data-schema-elsys} gezeigte Format. Auf diesem normalisierten Format, welches dann in \ac{JSON}-Syntax übergeben wird, können diverse Analysen zur \coo{} Konzentration in den Räumen, Belegung und ähnlichem durchgeführt werden. Die Distanz zwischen den einzeln übermittelten Geräten beträgt 3 Minuten, wobei jedoch anzumerken ist, dass die Geräte nicht synchron alle 3 Minuten Daten senden, sondern jeweils einen eigenen Senderythmus haben.

% Küche
% Raum eBusiness
% Raum Berlin
% Raum Böblingen
% Raum Dresden
% Raum Düsseldorf
% Raum Hannover
% Raum München
% Raum Wien

\subsection{Sensor Simulator}\label{chap:iotdevicesim}
Mithilfe von Node-Red und der Erweiterung (in der Plattform auch \enquote{Knoten} genannt) \enquote{iot-device-simulator-1-mqtt} des Github Nutzenden phyunsj\footnote{Siehe auch \url{https://github.com/phyunsj/iot-device-simulator-1-mqtt}} kann ein \ac{IoT} Sensor simuliert werden, welcher in beliebiger Frequenz (einstellbar im Knoten \enquote{Loop}) verschiedene Werte in vorspezifizierten Bereichen generiert. Gezeigt ist der Ablauf in \autoref{abb:DeviceSimFlow}. Diese Werte können entsprechend von den bereits vorgestellten Lösungen verarbeitet werden und bieten bei höherer Frequenz einen Anhaltspunkt, wie sich die Lösungen unter Last verhalten.
\begin{figure}[H]
\centering
\includegraphics[width=\textwidth]{graphics/Device-Simulator-Flow.png}
\caption{Node-Red Flow des Sensorsimulators}
\label{abb:DeviceSimFlow}
\end{figure}


