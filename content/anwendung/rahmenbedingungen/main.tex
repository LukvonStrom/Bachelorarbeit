\section{Rahmenbedingungen der Datenverabeitung}\label{chap:rahmendatenverarbeitung}
Aufgrund bereits getroffener Architekturentscheidungen durch das \ac{IoT} Team der SPIRIT/21 GmbH wird für die Kommunikation zwischen Geräten und dem verarbeitenden Backend das \ac{MQTT} Protokoll verwendet. \ac{MQTT} ist nach eigener Aussage ein extrem leichtgewichtiges Standardprotokoll für publish-/subscribe-basierten Nachrichtentransport.\footcite[Vgl.][]{o.V..2020} Und ist nach Analystenmeinung der de-facto Standard für \ac{IoT} Kommunikation.\footcite[Vgl.][]{Skerrett.25.10.2019}\nzitat \footcite[Vgl.][]{Cabe.17.04.2018} 

In nachfolgenden Anwendungsfällen wird angenommen, dass eine technologische Vorauswahl für unterstützende AWS Dienste erfolgt ist.
So könnte beispielsweise ein beliebiger, \ac{MQTT} kompatibler Messagebroker eingesetzt werden, es wird jedoch davon ausgegangen, dass der eingesetzte Message Broker aus Kompatibilitätsgründen zu den anderen \ac{AWS} Diensten \ac{IoT} Core ist.

Zusätzlich ist von einem Sendeintervall von 5 Minuten für die Sensoren auszugehen, welches gesetzt wurde um eine hohe Akkulebensdauer zu ermöglichen. Diese Einschränkung trifft aber nicht auf \autoref{chap:iotdevicesim} zu.

