\chapter{Schlussbetrachtung}\label{chapter:Schlussbetrachtung}
Folgend sollen die Ergebnisse der Arbeit zusammengefasst und kritisch reflektiert werden. Anschließend wird ein Ausblick gegeben.

\section{Zusammenfassung}\label{section:Zusammenfassung}
Das Ziel der Arbeit bestand darin, Konzepte für Referenzarchitekturen zur Verarbeitung von Zeitreihendaten zu konstruieren.
Zuerst wurden mittels einer Literaturrecherche wesentliche theoretische Grundlagen erläutert. Dabei wurde die Definition von Zeitreihendaten betrachtet, aber auch für die Analyse wichtige Zusammenhänge wie beispielsweise der abnehmende Wert von Daten über die Zeit. Ebenfalls wurden für Zeitreihendaten populäre Auswertungsarten vorgestellt. Mit den $lambda$, $kappa$ und \ac{OLAP} Mustern wurden populäre Vorlagen für Architekturen dargestellt, welche in den Referenzarchitekturen ebenfalls Anwendung fanden.

Folgend wurde, erneut dank einer Literaturrecherche, das Vorgehen der Arbeit erläutert. 
Dazu wurde sowohl das Vorgehen zur Konstruktion von Referenzarchitekturen als auch wesentliche Einflussfaktoren auf die Referenzarchitekturen beleuchtet. Für die Referenzarchitekturen wurden verschiedene Dekompositionssichten dargestellt, genauso wie die Systematik der Variationspunkte, die die Referenzarchitekturen besser anwendbar machen sollen.
Mittels der erläuterten Methodik der Interviews konnten diese geführt und im Anhang transkribiert dargestellt werden. 
Für den durchzuführenden Dienstvergleich wurden die theoretischen Grundlagen für die einzelnen Bewertungsaspekte dargestellt. Dazu gehörte mit der Gesamtkostenrechnung eine Methodik, die dank des selben Usecases Vergleiche der monatlichen Kosten ermöglichen soll. Als ebenfalls für den Vergleich relevant wurde ein Featurevergleich basierend auf den bereits vorgestellten Auswertungen und ein Vergleich des durch den Hersteller garantierten Dienstleistungsumfangs vorgestellt. Für die Priorisierung der vorliegenden Kriterien wurde eine Umfrage anberaumt, mit welcher wichtige Stakeholder die Reihenfolge und Gewichtung der Kriterien bestimmen konnten.

Darauf folgend wurden die Anwendungsfälle erläutert, die bereits Zeitreihendaten produzieren und welche spezifischen Anforderungen diese an die Referenzarchitekturen stellen. Dabei wurde neben den Rahmenbedingungen, die beispielsweise die Nutzung des Kommunikationsstandards \ac{MQTT} umfassten, das Raumklimamonitoring der SPIRIT/21 in Böblingen oder ein Sensorsimulator vorgestellt. Der Sensorsimulator kann dabei diverse Anwendungsfälle mit unterschiedlichsten Frequenzen, Messwerten und Anforderungen simulieren.

In der Dienstauswahl konnte mittels der Ergebnisse der durchgeführten Umfrage eine Kriterienpriorisierung abgeleitet werden. 
Aufbauend auf dieser Priorisierung wurden die Dienste in drei Gruppen verglichen: Echtzeit, Batch und Multimode. Dabei wurde ein Teil der Vergleiche im Anhang durchgeführt.
Die Dienstauswahl im Echtzeitvergleich wurde dabei durch eine Kombination von mehreren Diensten der Kinesis Familie angeführt.
Die Dienstauswahl im Batchvergleich hingegen wurde von Amazon Timestream angeführt. 
Da in der Kategorie Multimode mit \AWSIOT{} Analytics nur ein einzelner Dienst verglichen wurde, führte dieser den Vergleich an.
Im Vergleich zu Timestream, mit dem \AWSIOT{} Analytics konkurrierte, unterlag \AWSIOT{} Analytics durch eine geringere Punktezahl.

Dank der priorisierten Dienste konnten im Kapitel Modellierung zwei Referenzarchitekturen konzipiert werden.
Dazu wurden zuerst die Ergebnisse der Interviews dargestellt.
Aus den Interviews wurden wesentliche Aspekte der Referenzarchitekturen, wie beispielsweise die Dekompositionstiefe oder die Allgemeingültigkeit zur Konstruktion abgeleitet. 
Folgend wurde der Aufbau beider Referenzarchitekturen inklusive der Dekompositionen erläutert.
Nachfolgend wurden die Referenzarchitekturen mit verschiedenen Dekompositionssichten modelliert. Dazu wurden als Dekompositionen verschiedene Sichten dargstellt. Sowohl die Datenverarbeitungssequenz, als auch die Verteilungssicht und Bausteinsicht wurden dabei mit Variationspunkten versehen, um eine Anpassung auf die Anforderungen der instanziierenden Architektur zu ermöglichen. Ebenfalls wurden bei den Elementen der Referenzarchitekturen, die mit Code zu implementieren sind, Codebeispiele in \anhangref{anhang:echtzeit-codesample} und \anhangref{anhang:batch-codesample} in der Sprache JavaScript gelistet. In den Referenzarchitekturen wurde auch ein Abgleich mit den erhobenen Anforderungen durchgeführt, ein produktives Monitoringkonzept dargestellt und Randbedingungen sowie Einflussfaktoren gezeigt.
Im Folgenden Vergleich wurden die Architekturen im Bezug auf die Anwendbarkeit auf verschiedene Usecases verglichen. Dabei wurden die in \anhangref{anhang:interview-philipp-03.05.2021}, \anhangref{anhang:interview-viet-04.05.2021} und \anhangref{anhang:interview-jan-04.05.2021} transkribierten Gespräche ausgewertet und Vorschläge, die sich auf den Review-Versionspunkt bezogen, soweit möglich implementiert. Damit konnten die für eine Referenzarchitektur wichtigen Faktoren Anwendbarkeit und Akzeptanz durch Stakeholder der SPIRIT/21 gemessen werden.

Mittels der open-source vorliegenden (Siehe: \url{https://github.com/LukvonStrom/Bachelorarbeit}) Referenzarchitekturen, können künftig interne Usecases und Kundenusecases im Bereich Zeitreihenverarbeitung in \ac{AWS} leichter umgesetzt werden, da es ab sofort eine klar definierte Referenzarchitektur mit definierten Variationspunkten gibt. 


\section{Kritische Reflexion}\label{section:Kritische-Reflexion}

Im Rahmen der konstruierten Monitoringkonzepte fehlen konkrete Maßnahmen, die im Falle einer Fehlermeldung auszuführen sind. Dies resultiert aus dem reaktiven Incident-response Ansatz, den die SPIRIT/21 im Bezug auf Incidents mit \ac{AWS}-nativer Infrastruktur durchführt. Im Rahmen des Betriebs von Architekturen, die von einer der Referenzarchitekturen abgeleitet sind, sollte deshalb, wie in \autoref{section:Einsatzszenarien-der-Referenzarchitekture} beschrieben, Chaos Engineering praktiziert werden. Alternativ kann auf abnormale Metriken allein reaktiv und nicht proaktiv reagiert werden, was möglicherweise Vertragspflichten im Kundenfall verletzt.

Im Rahmen der Anforderungserhebung wurden mit den diversen Stakeholdern Interviews geführt. Bei diesen schien die Verständlichkeit der vorab zugesendeten Materialien nicht vollständig gegeben zu sein. Aus diesem Grund musste den Stakeholder teilweise erklärt werden, wie die Kriterien konkret zu verstehen sind, damit diese eine Bewertung vornehmen konnten. Dies hätte womöglich durch die Bereitstellung von mehr Kontext bereits vor den Interviews verhindert werden können.

Obwohl die Referenzarchitekturen in den wesentlichen Punkten mit den Vorschlägen des \ac{AWS} Well-Architected Frameworks (bzw. im spezifischen der Analytics Lens) übereinstimmen,\footnote{Siehe dazu auch: \anhangref{anhang:interview-philipp-03.05.2021}} wurden die Architekturen nicht an den Kriterien final gemessen. Dies liegt auch mit der Ambiguität von Kriterien wie \textit{Orchestrate ETL workflows} zusammen, bei denen die Erfüllung schwer gemessen werden kann.\footcite[Vgl.][6]{Ravirala.2020}

Aufgrund des hohen Bezuges auf \ac{AWS} Technologien und Dienste musste mit einigen Quellen gearbeitet werden, die von \ac{AWS} verfasst wurden oder von Personen, die mit \ac{AWS} affiliiert sind oder waren. Wo möglich und verfügbar, wurden kritische Positionen eingebunden.

\section{Ausblick}\label{section:Ausblick}
Wie im Vergleich der Referenzarchitekturen bereits erwähnt, wurde das Thema \textit{Fog computing} in der Arbeit ausgespart. Die Referenzarchitekturen könnten bei strikten Verbindungsanforderungen um \textit{Fog computing} unter Zuhilfenahme von \ac{AWS} Greengrass ergänzt werden. Dabei könnte ergänzender Code auf Greengrass in Form von Containern oder lokalen Lambdafunktionen Benachrichtigungen lokal versenden und beispielsweise Aktoren auslösen. Die konkrete Implementierung ist abhängig von der ausgewählten Referenzarchitektur, den Übertragungstechnologien und der Bereitschaft des Kunden, Gateways via \ac{AWS} verwalten zu lassen.

Ergänzend könnten in Zukunft dank Machine Learning genauere Vorhersagen basierend auf den bisherigen Daten erzeugt werden. Diese könnten ebenfalls für Planungs- und Entscheidungszwecke verwendet werden. So könnte beispielsweise \textit{predictive maintenance} realisiert werden, um defekte Sensoren oder Geräte frühzeitig auszutauschen oder zu warten. Dazu wäre es notwendig, das Machine Learning via Lambda oder respektive bei der Echtzeitreferenzarchitektur in Kinesis Data Analytics abzuwickeln. Aus der Integration von Machine Learning mit Zeitreihendaten in der Cloud könnte sich auch ein neues Offering der SPIRIT/21 in Kundenprojekten ergeben.

Anschließend zur Abgabe der Bachelorarbeit wird das GitHub Repository mit dem Quelltext der Bachelorarbeit veröffentlicht. Zusätzlich werden die Referenzarchitekturen an mehreren internen Stellen wie dem internen Wissensmanagement Confluence abgelegt, um einfache Zugänglichkeit zu gewährleisten. Durch Kommunikation in Regelmeetings mit einer hohen Zahl an Mitarbeitenden mit dem Tätigkeitsschwerpunkt Cloud soll folgend erreicht werden, dass die Mehrheit der Mitarbeitenden die Referenzarchitekturen einsetzt. 

Nach Fertigstellung der Arbeit ist vorgesehen, die Referenzarchitekturen mittels dem \ac{AWS} \ac{CDK}, der \ac{AWS}-nativen \ac{IaC} Lösung umzusetzen. Mittels der Umsetzung als \ac{IaC}-Konstrukt, kann die Infrastruktur schnell wiederwendet und angepasst werden. Zusätzlich erlaubt eine \ac{IaC} Lösung wie das \ac{AWS} \ac{CDK} das automatisches Ausrollen aller beteiligten Infrastrukturkomponenten. So ist die Wiederverwendbarkeit über die reinen Architekturmuster hinaus gegeben, da allein der Code der beteiligten Ressourcen angepasst werden muss. Ebenso können Variationspunkte im Code mit Verweis auf diese Arbeit vermerkt werden und direkt integriert werden.

Wie auch in \anhangref{anhang:interview-viet-04.05.2021} bestätigt, werden die Referenzarchitekturen in künftigen Kunden-Usecases genutzt. Damit kann sich der Nutzen für die Native Cloud Solution der SPIRIT/21 auch in der Praxis zeigen. Ebenfalls ist zu erwarten, dass die Referenzarchitekturen, wie in \anhangref{anhang:interview-philipp-03.05.2021} erwähnt, für interne Lösungen verwendet werden.