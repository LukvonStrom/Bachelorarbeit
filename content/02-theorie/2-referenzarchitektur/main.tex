\section{Referenzarchitektur}\label{theorie:referenzmodellierung}
Nach \citeauthor{Bass.2010} ist eine Referenzarchitektur ein spezialisiertes Referenzmodell, wie in \autoref{abb:RelationshipsReferenceModel} gezeigt, welches in Softwarearchitekturen instanziiert werden kann. \footcite[Vgl.][S.~17~f.]{Bass.2010} 

\begin{figure}[H]
\centering
\includegraphics[width=0.66\textwidth]{graphics/Relationships-reference-models.pdf}
\caption[Beziehungen zwischen Referenzmodellen, Architekturpatterns, Referenzarchitekturen und Softwarearchitekturen]{Beziehungen zwischen Referenzmodellen, Architekturpatterns, Referenzarchitekturen und Softwarearchitekturen\footnotemark}
\label{abb:RelationshipsReferenceModel}
\end{figure}
\footnotetext{Mit Änderungen entnommen aus: \cite[][18]{Bass.2010}}

In diesem Kapitel sollen deshalb die theoretische Definition des Referenzarchitekturbegriffs, sowie mögliche Vorgehensmodelle betrachtet werden.



\subsection{Referenzmodelle}

Gemäß des konstruktionsprozessorientierten Referenzmodellbegriffs von \citeauthor{vomBrocke.2003} ist ein Referenzmodell als solches zu erkennen, wenn der Gegenstand und/oder\footnote{Die Verwendung von \enquote{und/oder} wurde hier gewählt, da der Autor der Quelle das \enquote{oder} aus der boolschen Algebra gewählt hat um explizit beide Fälle einzuschliessen.} der Inhalt des Referenzmodells bei der Konstruktion des Gegenstandes und/oder des Inhaltes eines zu konstruierenden Anwendungsmodells wiederverwendet werden kann.\footcite[Vgl.][34]{vomBrocke.2003} Dabei hat ein Referenzmodell einen Empfehlungscharakter und stellt eine \enquote{best practice} dar.\footcite[Vgl.][31]{vomBrocke.2003} 

Ein Referenzmodell kann nach \citeauthor{vomBrocke.2003} nicht objektiv allgemeingültig sein und auch keinen objektiven Empfehlungscharakter haben, sondern muss subjektiv beurteilt werden.\footcite[Vgl. auch im Folgenden][31~f.]{vomBrocke.2003}  Dabei ist zumindest von den Interessensgruppen der Konstruierenden und der Nutzenden auszugehen, welche das Referenzmodell subjektiv unterschiedlich nach Allgemeingültigkeit und Empfehlungscharakter bewerten. Je nachdem welche Beurteilung höher gewichtet wird und früher einfließt, kann also entweder von der Situation ausgegangen werden, dass das Referenzmodell vom Konstruierenden zu einem solchen erklärt wird oder ein Modell, ob vom Konstruierenden beabsichtigt oder nicht, von den Nutzenden zu einem solchen erhoben wird.


\subsection{Referenzarchitektur}
Der IEEE Standard 1471-2000 definiert Architektur im Kontext von softwareintensiven Systemen wie folgt:
\enquote{The fundamental organization of a system embodied in its components, their relationships
to each other, and to the environment, and the principles guiding its design and evolution.} \footcite[][3]{IEEEComputerSociety.2000}. Als softwareintensives System kann jedes System gesehen werden, bei dem Software essentielle Einflüsse auf das Design, die Erstellung, das Deployment oder die Evolution des Systems hat.\footcite[Vgl.][1]{IEEEComputerSociety.2000}
Wird dieser Architekturbegriff auf bekannte Bereitstellungsmodi aus der Cloud, wie \ac{SaaS}, \ac{PaaS}, \ac{IaaS} oder \ac{FaaS} angewendet, wird klar, dass von einer Architektur im Sinne des IEEE Standards 1471-2000 ausgegangen werden kann, sobald Software involviert ist. Im Rahmen dieser Arbeit werden auch Dienste, die sich nach der NIST Cloud Definition unter \ac{SaaS} Dienste zählen lassen, behandelt. Als \ac{SaaS} Dienst gilt dabei jeder Dienst, bei dem Nutzende die unterliegende Infrastruktur nicht verwalten und die Applikation nur über limitierte Konfigurationen verwalten können. Sollte aber die Konfiguration mittels einer Programmiersprache bzw. Datenabfragesprache wie \ac{SQL} erfolgen, ist die Bedingung erfüllt, dass Software wesentliche Einflüsse auf das System hat.

\citeauthor{Gallagher.2000} definiert eine Referenzarchitektur als eine generalisierte Architektur mehrerer Endsysteme, die eine oder mehrere Domänen teilen.\footcite[Vgl. auch im Folgenden][3]{Gallagher.2000} Die Referenzarchitektur definiert nach Sicht des Autors dabei die gemeinsame Infrastruktur der Endsysteme und die Schnitstellen der Komponenten, die in den Endsystemen enthalten sein sollen. Dabei ist eine Referenzarchitektur zu instanziieren, um eine spezifische Softwarearchitektur zu erstellen. Gallagher definiert die Aufgaben einer Referenzarchitektur wie folgt: Zum einen werden übergreifende Funktionen und Konfigurationen generalisiert und extrahiert, und zum anderen wird eine kosteneffiziente und verlässliche Basis geschaffen, um Zielsysteme abzuleiten/zu instanziieren.\footcite[Vgl.][3]{Gallagher.2000}

\citeauthor{Trefke.2012} schränkt in seiner Definition die Instanziierung insoweit ein, dass individuelle Besonderheiten abstrahiert werden müssen, um eine Allgemeingültigkeit der Referenzarchitektur in einer speziellen Domäne zu erhalten.  \footcite[Vgl. auch im Folgenden][]{Trefke.2012} Zusätzlich fügt \citeauthor{Trefke.2012} der Referenzarchitektur als optionale Aufgaben die Definition von Leitlinien für die Verwendung, Evolution und Verantwortlichkeiten hinzu. Zurückgreifend auf \citeauthor{vomBrocke.2003} legt \citeauthor{Trefke.2012} fest, dass eine Referenzarchitektur als spezifischeres Referenzmodell seinen Empfehlungscharakter entweder durch Erfahrungen und hohe Nutzerakzeptanz oder durch Festsetzung von Erschaffenden erhält.

Nach \citeauthor{Angelov.2012} gibt es zwei Typen und damit verbundene Zielsetzungen der Referenzarchitektur: Die standardisierende Referenzarchitektur, welche darauf zielt eine Standardarchitektur für spezielle Anwendungsfälle zu schaffen, und die unterstützende/erleichternde Referenzarchitektur, welche Personen in Architekturrollen unterstützen sollen, ähnliche Probleme leichter zu lösen.\footcite[Vgl. auch im Folgenden][S.~422~ff.]{Angelov.2012} Nach \citeauthor{Angelov.2012} sind standardisierende Referenzarchitekturen nicht zur Verwendung von innovativen, also kaum getesteten oder noch nicht von Experten akzeptierten Elementen geeignet. Die unterstützenden/erleichternden Referenzarchitekturen hingegen können solche innovativen Elemente durchaus verwenden und auch eine Technologievorauswahl treffen.

Um einen möglichst hohen Nutzen stiften zu können, müssen die organisatorischen Rahmenbedingungen, in welchen ein Referenzmodell eingesetzt werden soll, analysiert werden.\footcite[Vgl.][]{vomBrocke.2004}
\citeauthor{Muller.2020} empfiehlt, eine Referenzarchitektur zur Generalisierung von vorhandenen Architekturen zu verwenden.\footcite[Vgl. auch im Folgenden][7]{Muller.2020} Für neue Technologien und Applikationen, die bislang in der Form kaum verwendet wurden, schlägt \citeauthor{Muller.2020} stattdessen ein inkrementelles Vorgehen vor. Das inkrementelle Vorgehen von \citeauthor{Muller.2020} beinhaltet dabei die Erstellung von Prototypen und Einholung von Feedback der Zielstakeholder. 

\subsection{Diskussion der Qualitätskriterien der Referenzarchitekturen}\label{chap:qualitycriteria}
\citeauthor{Muller.2020} schlägt sieben Qualitätskriterien vor, welche von einer guten Referenzarchitektur erfüllt werden sollten \englishquote{englisches Original geklammert}:\footcite[Vgl. auch im Folgenden][8]{Muller.2020}
\begin{enumerate}
\item Verständlichkeit für eine breite, heterogene Gruppe an Stakeholdern (Kunden, Projektmanager, Entwickler, etc.) \englishquote{understandable for a broad set of heterogeneous stakeholders}
\item Zugänglichkeit und Zugriff durch die Mehrheit der Organisation \englishquote{accessible and actually read/seen by majority of the organization}
\item Adressierung der Hauptprobleme der spezifischen Problemdomäne \englishquote{addresses the key issues of the specific domain}
\item Zufriedenstellende Qualität \englishquote{satisfactory quality}
\item akzeptabel \englishquote{acceptable}
\item \enquote{up-to-date} und wartbar \englishquote{up-to-date and maintainable}
\item wertschöpfend für den Betrieb \englishquote{adds value to the business}
\end{enumerate}
Die Erfüllung dieser Qualitätskriterien wird auf unterschiedliche Weise gemessen. Die Erfüllung der Kriterien 1, 3, 4 und 5 soll durch Interviews diverser Stakeholder der Cloud-Entwicklung gezeigt werden. Dabei wird auf einen fixierten Versionspunkt der Referenzarchitekturen referenziert. In den Interviews sollen die Stakeholder unformalisiert nach ihrem Eindruck zum Erfüllungsgrad der Qualitätskriterien befragt werden. Die Kriterien 2, 6 und 7 können durch einen Fachbeweis erfüllt werden. Dieses Vorgehen entspricht in Teilen dem inkrementellen Vorgehen von \citeauthor{Muller.2020}.

% Für die technischen Aspekte der Referenzarchitekturen definiert \ac{AWS} im Rahmen des sogenannten Well Architected Frameworks Themenfelder, die bei exzellenten Architekturen zu beachten sind. Für Datenanalysen gibt es die Analytics Lens, welche entsprechend für Referenzarchitekturen genauso Anwendung finden sollte. Kriterien sind dabei die Folgenden:\footcite[Vgl.][6]{Ravirala.2020}
% \begin{enumerate}
% \item Automatisierte Datenaufnahme \englishquote{Automate data ingestion}
% \item Gestaltung der Datenaufnahme für Fehler und Duplikate \englishquote{Design ingestion for failures and duplicates}
% \item Aufbewahrung der originalen Daten \englishquote{Preserve original source data}
% \item Beschreibung der Daten mit Metadaten \englishquote{Describe data with metadata}
% \item Beschreibung der Datenherkunft \englishquote{Establish data lineage}
% \item Nutzung des richtigen \ac{ETL} Tools \englishquote{Use the right ETL tool for the job}
% \item Orchestrierung von \ac{ETL} Workflows \englishquote{Orchestrate ETL workflows}
% \item Errichtung von angemessenen Speicherebenen \englishquote{Tier storage appropriately}
% \item Sicherung und Verwaltung der gesamten Analysepipeline \englishquote{Secure, protect, and manage your entire analytics pipeline}
% \item Gestaltung von skalierbaren und verlässlichen Pipelines \englishquote{Design for scalable and reliable analytics pipelines} 
% \end{enumerate}

% \TodoW{In Text einbetten, Erklärung welche passen}

\subsection{Vorgehensmodell}

\citeauthor{Schutte.1998} unterteilt, wie in \autoref{abb:VorgehensmodellReferenzmodellierung} gezeigt, Referenzmodellierung generell in vier Phasen.\footcite[Vgl. auch im Folgenden][184\psq]{Schutte.1998} 

Die erste Phase, die Problemdefinition ist mit der Darstellung des behandelten Problems in der Einleitung dieser Arbeit bereits vorgenommen worden. Laut \citeauthor{Schutte.1998} kann der Wirklichkeitszugang des erstellten Modells nur über Bilder erfolgen, welche entsprechend zu modellieren sind.\footcite[Vgl. auch im Folgenden][185\psq]{Schutte.1998} Da momentan keine Erfahrungen in den zu verwendenden Technologien für die Referenzmodellierung vorliegt, handelt es sich um eine Top-down Referenzmodellierung/Referenzarchitektur. 

Bei der Konstruktion des Referenzmodellrahmens wird durch das \enquote{Was} motiviert, welche Unternehmensspezifika zu beachten sind.\footcite[Vgl. auch im Folgenden][186]{Schutte.1998} In der vorliegenden Arbeit wäre beispielsweise Teil des Referenzmodellrahmens, dass cloudbasiert gearbeitet werden soll und dass die Zeitreihendaten vorerst vornehmlich von \ac{IoT} Anwendungsfällen stammen, was sich später dennoch ändern könnte.\footcite[Vgl. auch im Folgenden][187\psq]{Schutte.1998} 

In der \enquote{Wie} Phase, die dieses Kapitel behandeln soll, wird die Referenzarchitektur strukturiert und Darstellungsarten gezeigt, welche folgend angewendet werden können. 

Die eigentliche Konstruktion der Referenzmodelle oder hier der Referenzarchitektur erfolgt in \autoref{chap:ra-rt} und \autoref{chap:ra-batch}.

\begin{figure}[H]
\centering
\includegraphics[width=0.75\textwidth]{graphics/Vorgehen-Referenzmodellierung.pdf}
\caption[Vorgehensmodell Referenzmodellierung nach Schütte]{Vorgehensmodell Referenzmodellierung nach Schütte\footnotemark}
\label{abb:VorgehensmodellReferenzmodellierung}
\end{figure}
\footnotetext{Mit Änderungen entnommen aus: \cite[][185]{Schutte.1998}}

Nach \citeauthor{Muller.2020} hat eine Referenzarchitektur mehrere Dekompositionen\footnote{Mit Dekomposition ist hier die Zerlegung der Gesamtarchitektur auf eine vorspezifizierte Detailtiefe gemeint}, in beispielsweise eine funktionale, eine konstruktionsorientierte oder eine infrastrukturorientierte Komposition.\footcite[Vgl.][7]{Muller.2020} Diese Dekompositionsschichten lassen sich ebenfalls in bekannten Architekturframeworks wie arc42 finden. Aus diesem Grund soll ein Teil der Dekompositionen für die Referenzarchitektur verwendet werden. Dies deckt sich auch mit der Auffassung von \citeauthor{Schutte.1998} zur Referenzmodellierung, nach welchem Modellierung über Bilder erfolgen sollte.\footcite[Vgl.][185]{Schutte.1998}

\begin{figure}[H]
\centering
\includegraphics[height=0.23\textheight]{graphics/reference-architecture-details.pdf}
\caption[Gewünschter Detailgrad von Referenzarchitekturen nach \citeauthor{Muller.2020}]{Gewünschter Detailgrad von Referenzarchitekturen nach \citeauthor{Muller.2020}\footnotemark}
\label{abb:DetailgradMuller}
\end{figure}
\footnotetext{Entnommen aus: \cite[][11]{Muller.2020}}

Mehrere Dekompositionen machen insbesondere auch Sinn, da wie im Diagramm von \citeauthor{Muller.2020} - \autoref{abb:DetailgradMuller} - eine Addressierung von unterschiedlichen Aspekten wie Systemgestaltung, aber auch Stakeholder oder Kontext erfolgen sollte. Eine Adressierung der Stakeholder und des Kontextes ist insofern gegeben, dass, wie in \autoref{chap:requirements} gezeigt, Interviews durchgeführt und im Anhang dieser Arbeit transkribiert werden. An Stellen, an denen ein Einsatz von Teilen der Referenzarchitektur nontrivial scheint, kann durch Codebeispiele oder konkrete, kopierbare Codeausschnitte gezeigt werden, auf was speziell im technischen Bereich zu achten ist.

Zusätzlich sollen Dekompositionen unterschiedliche Aufgaben erfüllen. Speziell für die Darstellung der verwendeten Dienste von \ac{AWS} in den Referenzarchitekturen und dem Datenfluss soll die erste Stufe der Bausteinsicht des Architekturstandards arc42 verwendet werden. Das Konzept jener Bausteinsicht, also wie sie zu gestalten ist, findet sich in \autoref{abb:BausteinsichtStufe1}. In der finalen Umsetzung werden die offiziellen \ac{AWS} Icons die entsprechenden Services darstellen, die verwendet werden.

\begin{figure}[H]
\centering
\includegraphics[height=0.38\textheight]{graphics/Bausteinsicht.png}
\caption[Stufe 1 der Bausteinsicht in arc42]{Stufe 1 der Bausteinsicht in arc42\footnotemark}
\label{abb:BausteinsichtStufe1}
\end{figure}
\footnotetext{Mit Änderungen entnommen aus: \cite{Starke.o.J.}}

Zusätzlich zu der Bausteinsicht können, wie in \autoref{abb:Diagrammtypen} gezeigt, weitere Diagrammtypen eingesetzt werden, um verschiedene Dekompositionen darstellen zu können.

\begin{figure}[H]
\centering
\includegraphics[width=\textwidth]{graphics/Diagrammtypen.pdf}
\caption{Ergänzende Dekompositionen}
\label{abb:Diagrammtypen}
\end{figure}


Wie von \citeauthor{Muller.2020} vorgeschlagen und im vorherigen Unterkapitel erläutert, ist ein inkrementeller Ansatz unter Verwendung von Prototypen und kontinuierlichem Feedback der Zielstakeholder unerlässlich.\footcite[Vgl.][7]{Muller.2020} 

Sehr wichtig für eine Referenzarchitektur ist auch die Dokumentation, wie die Wiederverwendung zu handhaben ist. Ein möglicher Ansatz wäre dabei die gezielte Integration und Dokumentation von Variationspunkten, wie von \citeauthor{Webber.2001} vorgeschlagen.\footcite[Vgl.][24\psqq]{Webber.2001} Mittels der Variationspunkte kann eine statische Referenzarchitektur konstruiert werden, welche an spezifisch definierten Punkten angepasst werden muss, um einzigartige Architekturen zu erzeugen.\footcite[Vgl.][24]{Webber.2001} Dabei gibt es vier verschiedene Ansichten, aus denen Variationspunkte definiert werden können:\footcite[Vgl.][25\psq]{Webber.2001}
\begin{enumerate}
\item \label{view:first} Anforderungs-Variationspunktsicht \englishquote{Requirement-Variation-Point View}
\item \label{view:second} Komponenten-Variationspunktsicht \englishquote{Component-Variation-Point View}
\item \label{view:third} statische Variationspunktsicht \englishquote{Static-Variation-Point View}
\item \label{view:fourth} dynamische Variationspunktsicht\englishquote{Dynamic-Variation-Point View}
\end{enumerate}
Dabei sind für diese Arbeit, in der keine implementierungsnahe (im Sinne von Programmierung) Softwarearchitektur entworfen wird, hauptsächlich die anforderungsbasierte und die komponentenbasierte Variationspunktsicht wichtig. Die statische und dynamische Variationspunktsicht agieren stärker auf Implementierungsebene.\footcite[Vgl. auch im Folgenden][25\psq]{Webber.2001} Auf dieser können beispielsweise mittels objektorientierter Programmierung Klassen bereitgestellt werden, von welchen geerbt werden kann. Gleichzeitig kann die Verhaltensweise des Programms auch durch z.B. Callbacks oder Parameterisierung des Aufrufes verändert werden.


Variationspunkte können, wie in \autoref{abb:Variationspunkte} dargestellt innerhalb der verschiedenen, dargestellten Schichten wie folgt dargestellt werden:
\begin{figure}[H]
\centering
\includegraphics[height=1.33cm]{graphics/Variationpoints.pdf}
\caption{Darstellung Variationspunkte}
\label{abb:Variationspunkte}
\end{figure}
In den Architekturebenen werden entsprechend die Variationspunkte mit alphanumerischen Identifikationen versehen. So können gleiche Variationspunkte in mehreren Dekompositionen referenziert werden.


