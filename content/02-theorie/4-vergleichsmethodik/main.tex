\section{Vergleichsmethodik für die Dienstauswahl}\label{chap:vergleichsmethodik}

\citeauthor{Marz.2015}, die bereits die $\lambda$-Architektur geprägt haben, haben erwünschte Eigenschaften eines Big Data Systems festgelegt.\footcite[Vgl. auch im Folgenden][7\psqq]{Marz.2015} Da sich die Referenzarchitekturen ebenfalls mit großen Datenmengen arbeiten, wenn auch in anderer Weise können die Folgenden Kriterien übertragen werden:
\begin{enumerate}
\item Robustheit und Fehlertoleranz \englishquote{Robustness and fault tolerance} - 
Systeme sollen Herausforderungen, wie beispielsweise Paralellität, Datenduplikate oder technische Ausfälle verkraften. Zusätzlich ist Resilienz gegenüber menschlichen Fehlern wünschenswert, so dass händische Änderungen rückgängig gemacht werden können (also, dass beispielsweise Analysecode \enquote{immutable} ist).

\item Lese- und Schreibzugriffe mit niedriger Latenz \englishquote{Low latency reads and updates} - 
Lesezugriffe auf Daten sollen mit niedriger Latenz stattfinden. Wie bereits beschrieben, kann aufgrund der Messdistanz eine Aktualisierung von Daten durchaus längere Zeit benötigen, jedoch sollte ein Big Data System in der Lage sein, Datenaktualisierungen mit niedriger Latenz durchzuführen.

\item Skalierbarkeit \englishquote{Scalability} - 
Das Big Data System sollte durch transparente oder intransparente Provisionierung weiterer Ressourcen in der Lage sein, gleiche Performance in verschiedenen Belastungssituationen zu liefern. Dies deckt sich mit einem der Kernversprechen der Public Clouds nach NIST Definition \englishquote{rapid elasticity}.\footcite[Vgl.][2]{Mell.2011}

\item Generalisierung \englishquote{Generalization} - 
Ein Big Data System sollte in der Lage sein, verschiedene Anwendungen zu unterstützen. Da die Zielsetzung dieser Bachelorarbeit auf Zeitreihendaten aufbaut, welche wie in \autoref{chap:GrundlagenDatenanalyse} gezeigt, einen großen Einsatzspielraum haben, ist diese Bedingung bei ausreichender Generalisierung der Referenzarchitekturen erfüllt.

\item Erweiterbarkeit \englishquote{Extensibility} - 
Das zu gestaltende Big Data System soll erweiterbar sein und neue Funktionen oder Änderungen ohne größeren Aufwand ermöglichen.

\item Sofortige Abfrage \englishquote{Ad hoc queries} - 
Diverseste Abfragen sollen schnellstmöglich auf dem Datensatz der Big Data Anwendung möglich sein.

\item geringer Wartungsaufwand \englishquote{Minimal maintenance} - 
Eine Big data Anwendung soll wartbar bleiben, indem die Komplexität in den Kernkomponenten, welche nach Ansicht von \citeauthor{Marz.2015} zu erhöhtem Wartungsaufwand führt, möglichst gering ist.

\item Fehlertransparenz \englishquote{Debuggability} - 
Innerhalb eines Big Data Systems soll es möglich sein, nachzuverfolgen, wie Werte entstanden sind, um mögliche Fehler verfolgen zu können.
\end{enumerate}

Um die Vergleiche durchführen zu können, ist eine Priorisierung aller Kriterien notwendig. Zu diesem Zweck wird eine Umfrage mit mehreren Stakeholdern (mindestens $n=4$) durchgeführt. Diese Umfrage soll mindestens alle Stakeholder, die bereits interviewt wurden umfassen, kann aber auch zusätzliches technisches Personal des \ac{NCS} Teams inkludieren. Ausgehend von den priorisierten Kriterien kann eine gewichtete Bewertung der verglichenen Dienste erfolgen.


\subsection{Features des Dienstes}
Es soll auf die Mindestverfügbarkeit folgender, in \autoref{chap:auswertungsarten} beschriebenen Analysen überprüft werden:

\begin{enumerate}
\item Median und Quantile
\item Anomaliedetektion
\item Schwellwertüberschreitung
\item Trenderkennung/gleitender Durchschnitt
\end{enumerate}

Dabei ist zu dokumentieren, ob und in welcher Form der Dienst die Analysen unterstützt. Wenn Hilfsdienste wie Lambda benötigt werden, ist dies ebenfalls zu dokumentieren.

\subsection{Dienstleistungsumfang}
Es sind die Angaben des Herstellers zu bewerten und eventuelle Nutzungsfälle aufzulisten, bei denen der Dienst erfolgreich bei erhöhten Anforderungen (großes Datenvolumen, großer Durchsatz, etc.) eingesetzt wurde. Wenn vertragliche Garantien der Betreiber existieren, sollen diese ebenfalls dokumentiert werden. Dies kann sowohl Performancegarantien als auch Verfügbarkeiten oder andere Garantien, die im Rahmen des Bezugs der Dienstleistung erteilt wurden, umfassen.

\subsection{Gesamtkosten}
Um einen sinnvollen Kostenvergleich aufzustellen, sind die folgenden Annahmen zu treffen.


Es existieren 200 Geräte/Sensoren. Es geht eine Nachricht mit einem kB pro Minute pro Gerät ein (0,0432 GB/Gerät/Monat und 8,64 GB/Monat).

Es ist eine Vergleichsoperation auf einen Schwellwert auszuführen und, wo möglich, eine Zählung aller Schwellwertüberschreitungen der letzten drei Monate durchzuführen (historische Daten also mindestens 25,92 GB).  

Es ist die \ac{AWS}-Region Frankfurt (eu-central-1) mit Abrechnungswährung US-Dollar (Umrechnung in \texteuro{} erfolgt bei \ac{AWS} bei Abrechnung) zu wählen. Alternativ ist die Region Irland (eu-west-1) bei Nichtverfügbarkeit der Diensleistung in Frankfurt zu wählen. 

Dienste, die diese Analyse alleine nicht bewerkstelligen können, müssen unter zusätzlicher Verwendung von Rechendiensten wie Lambda oder \ac{EC2} angesetzt werden mit permanentem Speicher im \ac{S3}. 

Analysen, die individuell auslösbar sind, erfolgen alle 10 Minuten an Werktagen zwischen 9 und 17 Uhr, also monatlich 960mal. 

Es wird angenommen, dass der Schwellwert 5 mal pro Gerät pro Monat überschritten wird (1000 Überschreitungen insgesamt).
% Für die Zwischenspeicherung in \ac{S3}, wenn benötigt, wird folgendes Datenschema angenommen:
% \begin{listing}[H]
% \inputminted[frame=lines,breaklines=true]{json}{code/estimates/filtered-estimate.json}
% \caption[Beispiel JSON]{Beispiel \ac{JSON}}
% \label{listing:json}
% \end{listing}
Um Historien über 3 Monate bereitzustellen, sind entsprechend 3000 Einträge nötig, was eine Dateigröße von $\sim455,32$ kB ergibt.
\begin{table}[H]
\centering
\begin{tabular}{|l|l|l|}
\hline
Dimension & Preis(\$)/Einheit & Summe (\$) \\ \hline
Beispiel  & x/100.000 Datenpunkte & x  \\\hline
\begin{tabular}[c]{@{}l@{}}Beispiel 2\\(Speicher)\end{tabular}  & x/GB Speicher & x  \\\hline
Summe & \cellcolor[HTML]{EFEFEF} & \underline{x} \\ \hline
\end{tabular}
\caption{Kostenvergleich Schema}
\label{tab:kostenvergleich-schema}
\end{table}
Alle Abrechnungsdimensionen werden in der in \autoref{tab:kostenvergleich-schema} gezeigten Form dokumentiert.

