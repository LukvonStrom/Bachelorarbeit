\section{Vergleichsmethodik für die Produktauswahl}\label{chap:vergleichsmethodik}

\citeauthor{Marz.2015}, die bereits die $\lambda$-Architektur geprägt haben, haben folgende,erwünschte Eigenschaften eines Big Data Systems festegelegt:\footcite[Vgl.][7\psqq]{Marz.2015}
\begin{enumerate}
\item Robustness and fault tolerance

Systeme sollen Herausforderungen, wie beispielsweise Paralellität, Datenduplikate oder technische Ausfälle verkraften. Zusätzlich ist Resilienz gegenüber menschlichen Fehlern wünschenswert, so dass händische Änderungen rückgängig gemacht werden können (also beispielsweise Analysecode \enquote{immutable} ist).
\item Low latency reads and updates

Lesezugriffe auf Daten sollen mit niedriger Latenz stattfinden. Wie bereits beschrieben, kann aufgrund der Messdistanz eine Aktualisierung von Daten durchaus längere Zeit benötigen, jedoch sollte ein Big Data System in der Lage sein, Datenaktualisierungen mit niedriger Latenz durchzuführen.
\item Scalability

Das Big Data System sollte durch transparente oder intransparente Provisionierung weiterer Ressourcen in der Lage sein, gleiche Performance in verschiedenen Belastungssituationen zu liefern. Dies deckt sich mit einem der Kernversprechen der Public Clouds nach NIST Definition (\enquote{rapid elasticity}).\footcite[Vgl.][2]{Mell.2011}
\item Generalization

Ein Big Data System sollte in der Lage sein, verschiedene Anwendungen zu unterstützen. Da die Zielsetzung dieser Bachelorarbeit auf Zeitreihendaten aufbaut, welche wie in \autoref{chap:GrundlagenDatenanalyse} gezeigt, einen großen Einsatzspielraum haben, ist diese Bedingung bei ausreichender Generalisierung der Referenzarchitekturen erfüllt.
\item Extensibility

Das zu gestaltende Big Data System soll erweiterbar sein und neue Funktionen oder Änderungen ohne größeren Aufwand ermöglichen.
\item Ad hoc queries

Diverseste Abfragen sollen schnellstmöglich auf dem Datensatz der Big Data Anwendung möglich sein.
\item Minimal maintenance

Eine Big data Anwendung soll wartbar bleiben, indem Komplexität in den Kernkomponenten, welche nach Ansicht von \citeauthor{Marz.2015} zu erhöhtem Wartungsaufwand führt, möglichst gering ist.
\item Debuggability

Innerhalb eines Big Data Systems soll es möglich sein, nachzuverfolgen, wie Werte entstanden sind, um mögliche Fehler verfolgen zu können.
\end{enumerate}

% Kriterien von Lukas an ein Produkt:
% \begin{itemize}
% \item zufriedenstellende Integration mit AWS nativen Diensten (z.B. Monitoring)
% \item Möglichst serverless (wenig Maintenance Aufwand)
% \item skalierbarkeit auf 0 bis 10000
% \end{itemize}

% Anforderungen von Kunden
% \begin{itemize}
% \item Skalierbarkeit
% \item pay for what you use (no dead infra)
% \item transparente Fehler
% \end{itemize}

% \begin{itemize}
%     \item Übertragbarkeit (ggf. zwischen Clouds)
% \end{itemize}

Vielleicht noch:
Requirements:\footcite[Vgl.][]{Belur.2020}
\begin{itemize}
\item Unified experience for data ingestion and edge processing
\item Versatile out-of-the-box connectivity
\item Scalable stream processing with complex transformations
\item Operationalized business rules and ML models
\item Ability to handle unstructured data and schema drift:
\item Reusability of processing logic
\item Governance and lineage
\end{itemize}

\subsection{Featurevergleich}
Es soll auf die Mindestverfügbarkeit folgender Fähigkeiten überprüft werden:
\begin{itemize}
\item Auswertungen nach \autoref{chap:auswertungsarten}
\end{itemize}

\subsection{Performancevergleich}
Es muss mindestens die theoretische Verarbeitungslatenz verglichen werden

\subsection{Kostenvergleich}
Um einen sinnvollen Kostenvergleich aufzustellen, sind folgende Annahmen zu treffen:
Es existieren 200 Geräte/Sensoren. Es geht eine Nachricht mit einem kB pro Minute pro Gerät ein (0,0432 GB/Gerät/Monat und 8,64 GB/Monat).
Es ist eine Vergleichsoperation auf einen Schwellwert auszuführen und wo möglich eine Zählung aller Schwellwertüberschreitungen der letzten drei Monate durchzuführen (historische Daten also mindestens 25,92 GB).  Es ist die Region Frankfurt (eu-central-1) mit Abrechnungswährung US-Dollar (Umrechnung in € erfolgt bei \ac{AWS} bei Abrechnung) zu wählen alternativ ist die Region Irland (eu-west-1) bei Nichtverfügbarkeit der Diensleistung in Frankfurt zu wählen. Produkte, die diese Analyse alleine nicht bewerkstelligen können, müssen unter zusätzlicher Verwendung von Rechendiensten wie Lambda oder \ac{EC2} angesetzt werden mit permanentem Speicher, der historische Daten speichert (\ac{S3}). Analysen, wo individuell auslösbar erfolgen alle 10 Minuten an Werktagen zwischen 9 und 17 Uhr, also monatlich 960mal. Andernfalls wird angenommen, dass der Schwellwert 5 mal pro Gerät pro Monat überschritten wird (1000 Überschreitungen).
Für die Zwischenspeicherung in \ac{S3}, wenn benötigt, wird folgendes Datenschema angenommen:

\begin{listing}[H]
\inputminted[frame=lines,breaklines=true]{json}{code/estimates/filtered-estimate.json}
\caption[Beispiel JSON]{Beispiel \ac{JSON}}
\label{listing:json}
\end{listing}
Um Historien über 3 Monate bereitzustellen, sind entsprechend 3000 Einträge nötig, was eine Dateigröße von ~455,32 KB ergibt. Das Berechnungsskript ist im Anhang \ref{anhang:berechnung} abgedruckt.


\begin{table}[H]
\centering
\begin{tabular}{|l|l|l|}
\hline
Dimension & Preis/Einheit           & Summe \\ \hline
Beispiel  & x\$/100.000 Datenpunkte & x\$  \\\hline
\end{tabular}
\caption{Kostenvergleich Schema}
\label{tab:kostenvergleich-schema}
\end{table}
Es sind alle Abrechnungsdimensionen in der in \autoref{tab:kostenvergleich-schema} gezeigten Form zu dokumentieren.
