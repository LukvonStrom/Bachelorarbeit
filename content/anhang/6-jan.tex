\anhang{Gespräch Jan R.}\label{anhang:interview-jan-04.05.2021}
\begin{table}[H]
\begin{tabularx}{\textwidth}{|l|X|}
\hline
    Datum                  & 04.05.2021 \\ \hline
    Thema                  & Review Referenzarchitekturen \\ \hline
    \begin{tabular}[c]{@{}l@{}}Teilnehmende,\\ Position\end{tabular} & \begin{tabular}[c]{@{}l@{}}Lukas Fruntke, Verfasser\\ Jan R., Azure Architekt - \ac{NCS}\end{tabular}\\ \hline
\end{tabularx}
\end{table}
\newcommand{\JR}{\textbf{Jan R.:}~}

\LF Hallo Jan, herzlich willkommen zu dem Interview und vielen Dank, dass du die Referenzarchitekturen bewerten möchtest.

\JR Gerne.

\LF Ich beziehe mich momentan auf den GitHub Commit 1527391 - \url{https://github.com/LukvonStrom/Bachelorarbeit/commit/1527391}. Das ist der Stand, den wir uns gerade zusammen angeschaut haben. Ich würde gerne von dir wissen: sind die Referenzachitekturen für dich, wo du eher einen microsoft Azure lastigen Background hast, also deinen Fokus eher auf einer anderen Public Cloud hast, verständlich?

\JR Im Großen und ganzen ja, es ist verständlich. Mir fehlen natürlich so ein paar Fachbegriffe, da ich keine Übersetzung in Azure Dienste habe. An sich geht es aber. 

\LF Okay, würdest du also sagen, die Qualität der Referenzarchitektur stellt dich zufrieden? Und du würdest circa wissen, wie du die, wenn du jetzt etwas mit \ac{AWS} machen müsstest, anwenden könntest?

\JR Also ich kann mir mit der Referenzarchitektur schon vorstellen, was ich machen soll und von daher stellt es mich auch zufrieden. Das passt.

\LF Okay, also ich entnehme dem, dass die Referenzarchitektur für dich akzeptabel ist.

\JR Ja, auf jeden Fall.

\LF Hast du irgendwelche Kritikpunkte oder Sachen, die ich noch verbessern sollte in der Referenzarchitektur?

\JR Also ich tue mir noch ein bisschen schwer, die Zuordnung der Variationspunkte nachzuvollziehen. Aber das ist eher eine kosmetische Sache. Die Infos stehen an der richtigen Stelle, sie sagen das richtige aus. Von demher passt das.

\LF Okay, dem entnehme ich, dass du insgesamt zufrieden bist. Ich werde nochmal schauen, ob ich die Darstellung anpassen kann, gegenüber der Version, die wir uns gerade angeschaut haben. Herzlichen Dank für deine Meinung!

\JR Bitte, gerne.