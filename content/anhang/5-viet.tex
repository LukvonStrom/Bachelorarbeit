\anhang{Gespräch Trung Viet L.}\label{anhang:interview-viet-04.05.2021}
\begin{table}[H]
\begin{tabularx}{\textwidth}{|l|X|}
\hline
    Datum                  & 04.05.2021 \\ \hline
    Thema                  & Review Referenzarchitekturen \\ \hline
    \begin{tabular}[c]{@{}l@{}}Teilnehmende,\\ Position\end{tabular} & \begin{tabular}[c]{@{}l@{}}Lukas Fruntke, Verfasser\\ Trung Viet L., Head of Solution - \ac{NCS}\end{tabular}\\ \hline
\end{tabularx}
\end{table}
\newcommand{\TVL}{\textbf{Trung Viet L.:}~}

\LF Hallo Viet! Herzlich willkommen zum Review der Referenzarchitekturen, die wir uns gerade angeschaut haben. Ich beziehe mich auf den Commit 1527391 - \url{https://github.com/LukvonStrom/Bachelorarbeit/commit/1527391}. Du bist als Head of Solution der Cloud Native Solution ja mit dafür verantwortlich, Architekturmuster festzulegen und die Wiederzuverwenden und auch die Cloud Entwicklung mit zu steuern. Siehst du für dich die Verständlichkeit der Referenzarchitekturen, die ich gestaltet habe, als gegeben an?

\TVL Ja, sehe ich.

\LF Okay, das heißt für dich sind da keine Fragen in irgendeiner Weise offen?

\TVL Nein überhaupt nicht. Die \ac{AWS}-Services sind mir geläufig, ich kenne die alle mit ihren Funktionalitäten. So wie die miteinander verknüpft wurden, so passt das auch.

\LF Heißt das, du würdest für dich und deine Solution die Referenzarchitektur akzeptieren? Also macht das auch Sinn für dich, abgesehen davon, dass du es verstehst?

\TVL Ich finde die Referenzarchitekturen sogar sehr gut. Die haben einen Bezug auf den serverless Bereich und haben einen guten Vorteil im Kosten/Nutzen Verhältnis. Deine Referenzarchitekturen sind sehr gute Patterns für die \ac{AWS} Cloud.

\LF Das heißt, wenn du in Zukunft Opportunities im Bereich Zeitreihenverabreitung bekommen würdest, würdest du die Patterns anwenden?

\TVL Genau. Aufgrund der guten Skalierbarkeit würde ich die Referenzarchitekturen, beziehungsweise die, die besser passt eins zu eins übernehmen.

\LF Das heißt, insgesamt stellt dich die Qualität der Referenzarchitektur, wie sie gestaltet und dokumentiert ist zufrieden? Oder hättest du noch irgendwas, dass du gerne verbessern würdest.

\TVL Die Dokumentation ist gut. Ich hätte mir vielleicht die weitere Unterteilung der Bilder in Segmente geünscht. Aber sonst ist es sehr verständlich.

\LF Okay, jetzt sind das ja schon verschiedene Dekompositionssichten, wie hättest du dir das dann noch extra vorgestellt?

\TVL Ich hätte mir gerne in den Dekompositionssichten an manchen Stellen eine Separation gewünscht.

\LF Okay, aber für dich wäre das jetzt auch okay wie es momentan ist?

\TVL Ja. Es ist ok, wie es gerade ist.

\LF Die Bachelorarbeit wird ja open-source sein. Da werden auch die .drawio Dateien dabei sein, die wir zum Austausch von Architekturskizzen benutzen. Wäre es für dich auch in Ordnung die draw.io Dateien zu benutzen, um die Diagramme gegebenenfalls zu ändern/anzuschauen?

\TVL Ja klar.

\LF Damit wäre es dann adressiert, oder?

\TVL Ja.

\LF Okay, ich entnehme aus dem Gespräch, dass du an sich zufrieden bist und die Referenzarchitekturen auch bei künftigen Kundenproblemen anwenden möchtest?

\TVL Voll und ganz, ja.

\LF Perfekt, dann vielen Dank für deine Zeit.

\TVL Sehr gerne.