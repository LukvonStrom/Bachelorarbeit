\anhang{AWS IoT Analytics} \label{anhang:vergleich-iot-analytics}

\anhangteil{Features des Dienstes}
Der \ac{SQL}-Analyseteil von IoT Analytics basiert wie Athena auf Presto und unterstützt die selben Funktionen und Operatoren.\footcite[Vgl.][]{AmazonWebServicesInc..o.J.au} Entsprechend sind die Anforderungen im selben Grad erfüllt wie bei Athena. Da \AWSIOT{} Analytics aber über die selbst programmierbaren Notebooks verfügt, lassen sich Verbesserungen an den Ansätzen mittels einer eigenen, angepassten Implementierung einzelner Analysen machen.
Da die Notebooks unter Python laufen können und ebenfalls pandas einbinden können, sind die selben Features wie bei \autoref{chap:vergleich-lambda} gegeben.

\anhangteil{Dienstleistungsumfang}
Die \ac{SLA} von \AWSIOT{} Analytics garantiert einzig die 99,9\% Verfügbarkeit im Monat.\footcite[Vgl.][]{AmazonWebServicesInc..2019f} Es gibt auch \textit{soft-limits}, welche auf Anfrage hochgesetzt werden können, welche einen Durchsatz von maximal 100.000 Nachrichten pro Sekunde vorsehen.\footcite[Vgl][]{AmazonWebServicesInc..o.J.av} Einzig die Performance der Analytics Compute Units ist durch vertikale Skalierung, also eine größere Buchung von Compute Units durch Nutzende beeinflussbar.

Bei \AWSIOT{} Analytics ist laut \ac{AWS} mit einer Latenz von Minuten oder Sekunden zu rechnen, was hoch ist im Vergleich zu Kinesis, wo mit Sekunden oder Milisekunden Latenz zu rechnen ist.\footcite[Vgl.][]{AmazonWebServicesInc..o.J.ax}