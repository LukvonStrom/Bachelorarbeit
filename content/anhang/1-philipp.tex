\interview{Philip A.}{Cloud Solution Architekt}{NCS}{Initiales Anforderunginterview}{22.03.2021}{philipp}

\newcommand{\LF}{\\ \textbf{Lukas F.:}~}
\newcommand{\PA}{\\ \textbf{Philipp A.:}~}

\textbf{Lukas F.:} Herzlich willkommen zum Interview und vielen Dank, dass du dich als Interviewpartner bereit gestellt hast
\PA	 Klar, bin ja da höchst involviert, insofern ist das komplett logisch
\LF	Ich habe direkt eine Frage an dich: Was ist deine Rolle innerhalb der SPIRIT und was ist deine Rolle im spezifischen mit Perspektive auf die Referenzarchitekturen, die es zu entwickeln gilt?
\PA	Ich sitze in der Spirit auf dem Posten des Cloud Solution Architects speziell für AWS und die entsprechende Rolle fülle ich eben auch in der Native Cloud Solution aus, das heißt, ich bin für Software und Infrastruktur Struktur Architekturen in dem Projekt/der Solution verantwortlich und kümmere mich darum, dass die Implementierung so gut wie möglich voran gehen kann, ohne Architektur Probleme zu verursachen. Ansonsten berate ich die Entwicklung  bei technischen Fragen und Implementierungsfragen, die auftreten
\LF	Wenn wir jetzt speziell Richtung Referenzarchitektur schauen, würdest du dich dann eher als Nutzender sehen oder eher als jemand, der zwar irgendwo einen \enquote{Stake} darin hat, dass es gut läuft, dass es eine gute der Referenz Architektur wird, aber jetzt konkret nicht die anwenden würde?
\PA	 Ich sehe mich auf beiden Seiten, sowohl als Nutzenden, weil ich würde natürlich versuchen auf Basis der Referenzarchitekturen unsere serverless solutions zu designen, also die für uns und unsere Kunden, aber ich glaube auf der anderen seite auch, dass ich mitwirke an der Ausarbeitung von solchen Referenzarchitekturen
\LF	 Verstehe, das ist schonmal aufschlussreich. Insgesamt, wir reden ja über Referenzarchitekturen, wo siehst du denn Anwendungsgebiete? Gibt es konkrete Customercases, wo wir jetzt auf Zeitreihendaten speziell schauen, oder gibt es da irgendwelche Gebiete, wo du sagst: \enquote{Oh, da könnte es besonders relevant sein}?
\PA	 Ja natürlich. Der größte/aktuellste Fall bei uns sind tatsächlich Sensordaten und Messdaten im \ac{IIoT} Umfeld, bei denen wir solche Zeitreihendaten immer haben. Aber auch sämtliche Monitoring Daten, die von cloud basierten, Metriken abgezogen werden und da gibt es viele.
\LF	 Das heißt also so wie ich es verstehe sowohl interne Anwendungsfälle als auch jetzt Kundenfälle, da ja die \ac{IIoT} Sachen hier involviert sind
\PA	 Nochmal die Frage?
\LF	So wie ich es verstehe, sowohl interne Anwendungsfälle mit Monitoring als auch Sachen, die für Kunden jetzt direkt relevant sind, wie \ac{IIoT}-/Sensordaten
\PA	 ja
\LF	 Das wären denke ich so die Bereiche, in denen man die Referenz Architekturen anwenden würde/ spezialisieren würde.
\PA	 Da wir uns, in der SPIRIT eben auch mit Applikation Management befassen. Ist es eben ein internes Thema. Weil wir eben selber sehen wollen, wie wir möglichst effizient Architekturen zum Betrieb, also auch zum Monitoring aufbauen können und vom Kunden eben auch, weil wir öfters Anfragen kriegen zum Thema \ac{IIoT} Umsetzung/ \ac{IIoT}-Implementierung und eben dann natürlich auch die entsprechenden Folgen davon, nämlich die Auswirkung der Daten.
\LF	Insgesamt, wenn wir an Referenzarchitekturen denken, ich rede davon gleich mal im Plural, weil es aus meiner Sicht schon mal mindestens zwei geben muss, als Artefakt meiner Bachelorarbeit. Wie kompatibel sollen die denn zueinander sein, wenn wir uns vorstellen, es gibt zum Beispiel eine für die Echtzeit Verarbeitung und eine für die Batch-Verarbeitung von Daten müssen die austauschbar sein, also von derselben Quelle zum Beispiel gefüttert werden? Oder ist es da okay jeweils recht spezifische Pipelines zu bauen, die  dann programmatisch eingebunden werden müssen, relativ nah am Datenerzeuger?
\PA	 Ich finde, das ist eine Frage die die Arbeit beantworten sollte, weil ich kann schwer abschätzen aus meiner aktuellen Position, ob es sinnvoller ist individuell, also wenn man dem Erzeuger nahestehend her geht  und  die Daten sozusagen dort auf ein Format zu bringen, die sich einfügen lassen oder ob es sinnvoller ist, die Daten vom Erzeuger so zu nehmen und dann in der Referenzarchitektur zu normalisieren und anzugleichen.
\LF	 Ich hatte jetzt nicht unbedingt an Daten direkt gedacht, sondern mehr an die Infrastruktur, die kompatibel sein muss. So dass es am Anfang eine Schnittstelle zum Beispiel geben würde, was eine Art Plug and Play mit beiden Ansätzen ermöglichen würde oder einfach das nur einen Ansatz quasi mit mit einer Customization sozusagen funktioniert
\PA	Ach so. Zu wünschen wäre es natürlich, wenn wir hinterher nur eine Hauptarchitektur hätten oder möglichst wenig Anpassungen an die Architekturen vornehmen müssen, um die Sachen zu switchen, ist natürlich aus der Architektenbrille die schönere Sache, auf der anderen Seite glaube ich tatsächlich das wenn man sich einmal festgelegt hat auf eine von beiden Arten, das man dann eh nicht mehr switchen wird, das heißt, wenn die Architekturen von sich aus schon unterschiedlich sind. Da muss man sich eben vorher klar werden, welchen Weg man gehen will. Es ist eigentlich völlig valide zu sagen wenn ich weiß, dass ich Batch Verarbeitung mache, dann habe ich auch genau diese Architektur. Die Anforderung, dass man wechseln muss, ist, glaube ich tatsächlich zu vernachlässigen, interessanter ist es eher, wenn man sich sicher ist oder wenn man sagt ich brauche beides ich brauche sowohl batch verarbeitung als auch Echtzeit verarbeitung ob es sich da nicht lohnt her zu gehen und sagen okay, wir haben eine referenz architektur die beides inkooperiert.
\LF	 Zu meiner nächsten Frage: Ich habe dir im vorfeld vom interviews zwei listen zugesendet, wo ich gerne deine meinung hören würde, wie du die jeweils priorisieren würdest. Zum einen sind das die Qualitätskriterien von Referenzarchitekturen und zum anderen die datenbezognenen Entscheidertypen. Fangen wir am besten mit den Qualitätskriterien an.
\PA	 Genau
\LF	Da hat es ja sieben. Welche würdest du denn relativ weit oben aus deiner position raus positionieren? Und welche sind von eher nachrangiger Relevanz?
\PA	Ich picke von den sieben mal drei raus und vielleicht kannst du mir zu dem einen oder anderen Punkt noch ein bisschen was erläutern. Der Punkt fünf \enquote{akzeptabel}, da verstehe ich nicht so wirklich, was du damit meinst. Beim  Thema \enquote{wertschöpfend für den Betrieb} versteh ich jetzt auch nicht so, was du damit meinst im vergleich zur Adressierung der Hauptprobleme. Kannst du das nochmal kurz erläutern?
\LF	\enquote{Wertschöpfend für den Betrieb}: So wie ich den Autor verstehe, mein das, das man einen Wert daraus hat, die Referenzarchitekturen anzuwenden und es sich nicht lohnt von \enquote{sratch} anzufangen.
\PA	 Okay. Wie sieht es mit akzeptabel aus?
\LF	Beim Kriterium \enquote{akzeptabel} verstehe ich den Autor so, dass man sich als jemand mit Fachkenntnissen im Prinzip das anschaut und sagt ja, das kann ich akzeptieren mit meinen fachkenntnissen und das ist jetzt nicht völlig an den Haaren herbeigezogen. Es ist quasi \enquote{reasonable}.
\PA	Das ist ja hoffentlich jeder Architektur, wenn man das nicht mal voraussetzen kann, dann muss man eigentlich punkt fünf als ersten nehmen es muss natürlich logisch, also \enquote{reasonable} sein. Das nächste muss sein das es wertschöpfend ist in irgendeiner weise, denn man macht nichts was nicht einen Mehrwert darstellt. Wo ich Schwierigkeiten habe, ist mit \enquote{reasonable}  weil \enquote{Adressierung der Hauptprobleme} hängt natürlich mit dem\enquote{reasonable} zusammen. Damit eine Referenzarchitektur wertschöpfend ist, muss sie aber auch verständlich sein. Jetzt ist die Frage, wie breit verständlichkeit gehen muss. Hier steht eine breite, heterogene Gruppe. So weit würde ich jetzt nicht gehen. Aber es muss verständlich sein für diejenigen, die die Referenzarchitektur anwenden müssen und da tatsächlich relativ einfach verständlich. Aber wenn es andere nicht gegeben, ist das sie \enquote{reasonable} ist, oder das die Probleme der Domäne nicht abgebildet werden, hilft sowieso alles nix.
\LF	 Das heißt du würdest tendentiell das es akzeptabel ist und die Probleme addressiert vorne anstellen, gefolgt von der Verständlichkeit
\PA	 Nein, wertschöpfend ist tatsächlich noch Nummer zwei, vielleicht sogar Nummer eins. Wenn es nicht wert schöpfend ist, dann brauche ich das nicht zu machen, dann habe ich nur Papierkrieg. Es muss einen Mehrwert für den Betrieb geben, das ist eigentlich Nummer eins, unser zweites ist, dass es akzeptabel sein muss, in Kombination mit der Problemlösung der Domäne. Dann kommen wir zu Thema Verständlichkeit.
\LF	 Verstehe. Wenn wir übergehen zu den Datenentscheider- oder -nutzungstypen gibt es ja drei Stück.  Die taktischen, die operativen und die strategischen Entscheider. Wo siehst du denn uns am ehesten? Man muss sich ja überlegen letztendlich ist das ja die Basis dafür, die wir die Daten analysieren wollen, also in welchen entscheidungs Horizont wir agieren und mit welcher Dringlichkeit wir neue Daten brauchen, um Entscheidungen abzuleiten.
\PA	 Was ist denn wir? Das hängt natürlich davon ab, was das für Daten sind und was der Ziel der Daten sind, beziehungsweise was das Ziel ist. Ist das Monitoring, dann habe ich da natürlich erst mal eine sehr kurzfristige Datenlage, die ich bewerten muss. Also keine Ahnung der Speicherplatz volläuft ist das wichtiger. Wenn ich irgendwelche \ac{IIoT} Daten habe, kann es durchaus sein, dass das eher langfristige Informationen sind. Wenn ich die Temperatur messe oder den CO2 Gehalt messe, habe ich auch durchaus Interesse an der Langfristigkeit der Daten.
\LF	 Verstehe also tendentiell würdest du wenn es um Monitoring Daten geht eher dem taktischen Entscheidertyp folgen, der recht früh betrachtet wenn sich was ändert und seine Entscheidung im Zweifelsfall anpasst. \ac{IIoT} siehst du also eher Richtung operative Entscheider?
\PA	Auch da hängt es wieder von der Art der Daten ab. Wenn es ein IIoT Sensor ist, der irgendwie messen soll, ob es z.B. ein Feuer gibt, dann ist es natürlich auch eine taktische Entscheidung, dann den Feueralarm zu betätigen. Wenn es aber Daten sind, die das Wetter beobachten, dann ist es vielleicht interessanter als strategischer Entscheider ranzugehen. Es ist sehr Datenbezogen. Beim Monitoring vielleicht ein bisschen weniger. Auch aus Monitoringdaten kann ich natürlich Sachen ziehen wenn ich nach einem halben Jahr Monitoring Daten sehe, in welchen Intervallen meine Systeme besonders ausgelastet sind.  Davon können natürlich auch Entscheidungen abgeleitet werden. Die meisten Informationen beim Monitoring sind aber tatsächlich kurzfristige. Und bei \ac{IIoT} kann ich das kann ich das gar nicht einschätzen, weil da bin ich nicht so tief drin und die Systeme von \ac{IIoT} sind so mannigfaltig. Das ist ja keine Domäne an sich, sondern es ist ja eher eine Infrastruktur, die auf verschiedene Domänen anwendbar ist, je nachdem, was das für Sensordaten sind oder auch in welchen Intervallen die abgefragt werden. Wenn es Sensoren gibt,die jede halbe Stunde Daten melden, dann ist die Echtzeit Entscheidung eher sekundär. Wenn es aber Daten sind, wo halt alle zehn Sekunden was anfällt ist das eher interessanter für Echtzeitentscheidungen
\LF	 Also du meinst es kommt wesentlich auf die Messdistanz der Sensordaten an?
\PA	 auf jeden Fall
\LF	 Verstehe. Zum zweiten Teil des Interviews: Ich habe ein Dimensionsmodell konstruiert für Referenzarchitekturen, wo es jetzt um die subjektive Allgemeingültigkeit, die Anwendbarkeit, und die Dekompositions Tiefe gehen soll. Jetzt wäre meine Frage jeweils, wie du drei Punkte einschätzt auf einer Skala von Null bis fünf und wieso. Wie wichtig ist dir subjektive Allgemeingültigkeit, wie tief muss eine Dekomposition stattfinden, damit Referenz Architekturen gut sind und wie konkret oder abstrakt darf die Referenzarchitektur sein, dass sie nutzenstiftend für viele Anwendungfälle ist, aber trotzdem eingesetzt werden kann.
\PA	 Die Allgemeingültigkeit und die Anwendbarkeit hängen stark vom Teilnehmerkreis ab, also wie groß sehen wir den Teilnehmerkreis, der Leute die mit dieser Referenzarchitektur arbeiten sollen? Je größer der ist, desto größer muss natürlich auch die Allgemeingültigkeit sein. Wenn wir das in dem relativ engen Rahmen sehen, auf Abteilungsebene oder vielleicht bisschen größer, dann können diese beiden Punkte relativ eng gefasst sein. Bei der Dekompositionstiefe , das ist aus meiner Sicht eine Fleißarbeit. Je detaillierter die Dekompositionstiefe ist, desto einfacher ist es, vermutlich die Referenz Architektur umzusetzen in eine echte Architektur, weil man da schon viele Hilfestellungen, Beispiele und sowas hat. Erfahrene Architekten können dann auch die Dekompositionstiefe wählen, die sie brauchen. Die Dekompositionstiefe ist bei der Anwendbarkeit Teil des Kontext. Da ist es eben die Frage, wie groß die Bandbreite ist, wenn wir sagen, wir haben genau diese zwei Usecases, nämlich Monitoring und \ac{IIoT}, dann kann die schon relativ konkret sein. Wenn wir sagen, wir haben vor, das irgendwie über die komplette Organisation und Firma zu stülpen, dann ist es halt sehr abstrakt, ich halte nicht so viel davon, Dinge zu abstrakt zu machen, weil sie dann tatsächlich oft nicht verstanden werden und auch nicht benutzt werden. Je abstrakter ich Sachen mache, desto geringere Dekompositionstiefe habe ich normalerweise. Insofern würde ich die das Thema Allgemeingültigkeit er so im mittleren und unteren Bereich sehen, genauso wie die Anwendbarkeit und Dekompositionstiefe so tief wie möglich.
\LF	Wenn du das auf einer Skala, wo null das schlechteste/geringst abstrakteste etc. ist und fünf die Vollausprägung, also sehr spezifische Allgemeingültigkeit beispielsweise, wo würdest du das dann jeweils sehen?
\PA	 Dann würde ich sagen, die subjektive Allgemeingültigkeit sollte irgendwo bei drei bis vier liegen, genauso die Anwendbarkeit im Kontext und die Dekomposition sollte sehr, sehr tief sein bei fünf.
\LF	 Herzlichen Dank, ich denke das ist für das erste Interview schon recht viel, was ich für meine Arbeit mitnehmen kann. Gerne würde ich mit dir ein weiteres Interview zum Abschluss der Arbeit machen, bei der wir die jetzt erarbeiteten Kriterien auf meine Artefakte anwenden und den Erfüllungsgrad messen.
\PA	 Können wir so machen.
\LF	Gut, dann vielen Dank für das Interview und deine Zeit.
\PA	 Gerne.