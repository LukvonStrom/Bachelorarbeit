\section{Theorie der Anforderungserhebung}

Gemäß der in \autoref{chap:qualitycriteria} definierten Qualitätskriterien und der von \citeauthor{vomBrocke.2003} aufgestellten Allgemeingültigkeitskriterien kann ein Referenzmodell und damit auch eine Referenzarchitektur nicht objektiv allgemeingültig sein. Zusätzlich ergeben sich als weitere wichtige Eingaben zur Konstruktion eines Referenzmodells die Dekompositionstiefe und die Anwendbarkeit. Zusammen lassen sich diese Dimensionen als Kiviat Diagramm\footcite[Vgl.][33\psqq]{Kolence.1973}, wie in \autoref{abb:Dimensionen} gezeigt, abbilden. 

\begin{figure}[H]
\centering
%\outertolerance{4}{4}{4}%defines the outer toleranceband for next spider
%\innertolerance{1}{1}{1}%defines the inner toleranceband for next spider
\spider{0}{0}{0}%draws the spider diagram with coordinates
\caption{Referenzarchitekturdimensionen}
\label{abb:Dimensionen}
\end{figure}

\Todo{Interviemethodik systematisieren Literatur}

Ziel einer Referenzarchitektur sollte also sein, das optimale Verhältnis der drei Dimensionen für die Zielstakeholder zu finden und daraus eine Referenzarchitektur zu erstellen. Um dieses Ziel zu erreichen, sind Interviews mit Zielstakeholdern der SPIRIT/21 zu führen, welche dem Interviewleitfaden in \autoref{tab:intervieleitfaden} folgen. Im referenzierten Leitfaden stehen dabei Fragen mit einem F-Präfix für allgemeine Fragen und Fragen mit einem D-Präfix für Fragen in Bezug auf die drei Dimensionen. Kombiniert mit mindestens einem Review der Referenzmodelle werden die Modelle zu einem nutzenstiftenden Artefakt.



\begin{table}[H]
\centering
\begin{tabular}{|l|l|}
\hline
ID & Beschreibung \\ \hline
\multicolumn{2}{|c|}{\cellcolor[HTML]{ECF4FF}Allgemeine Fragen} \\ \hline
F1 & Rolle innerhalb der SPIRIT/21 \\ \hline
F2 & Anwendungsgebiete der Referenzarchitekturen \\ \hline
F3 & Kompatibilität der Referenzarchitekturen zueinander? \\ \hline
\multicolumn{2}{|c|}{\cellcolor[HTML]{ECF4FF}Priorisierungen} \\ \hline
P1 & Priorisierung der Qualitätskriterien (siehe \autoref{chap:qualitycriteria}) \\ \hline
P2 & Priorisierung der Datennutzungstypen (siehe \autoref{chap:GrundlagenDatenanalyse}) \\ \hline
\multicolumn{2}{|c|}{\cellcolor[HTML]{ECF4FF}Dimensionen der Referenzarchitekturen} \\ \hline
D1 & Anforderungen an Anwendung der Referenzarchitekturen \\ \hline
D2 & Anforderungen an Allgemeingültigkeit der Referenzarchitekturen \\ \hline
D3 & Dekompositionstiefe der Referenzarchitekturen \\ \hline
\end{tabular}
\caption{Interviewleitfaden für Schlüsselstakeholder}
\label{tab:intervieleitfaden}
\end{table}


