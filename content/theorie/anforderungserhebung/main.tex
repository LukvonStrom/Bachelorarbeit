\section{Theorie der Anforderungserhebung}
\Todo{Rücksprache am 17.03.}
Gemäß der in \autoref{chap:qualitycriteria} definierten Qualitätskriterien und der von \citeauthor{vomBrocke.2003} aufgestellten Allgemeingültigkeitskriterien kann ein Referenzmodell und damit auch eine Referenzarchitektur nicht objektiv allgemeingültig sein. Ziel einer Referenzarchitektur sollte also sein, für die Zielstakeholder eine möglichst hohe subjektive Allgemeingültigkeit zu erreichen, um Nutzen zu stiften. Um dieses Ziel zu erreichen, sind Interviews mit Zielstakeholdern der SPIRIT/21 zu führen, welche dem Interviewleitfaden in \autoref{tab:intervieleitfaden} folgen. Kombiniert mit Reviews der entstehenden Artefakte, also der Referenzmodelle werden diese zu einem nutzenstiftenden Artefakt.

\begin{table}[H]
\centering
\begin{tabular}{|l|l|}
\hline
ID & Beschreibung \\ \hline
A1 & Rolle innerhalb der SPIRIT/21 \\ \hline
A2 & Anwendungsgebiete der Referenzarchitekturen \\ \hline
A3 & Anforderungen an Anpassbarkeit einer RA \\ \hline
A4 & Anforderungen an Darstellung einer RA \\ \hline
A5 & Anforderungen an Allgemeingültigkeit einer RA \\ \hline
A6 & Wie kompatibel sollen die Referenzarchitekturen zueinander sein \\ \hline
\end{tabular}
\caption{Interviewleitfaden für Schlüsselstakeholder}
\label{tab:intervieleitfaden}
\end{table}