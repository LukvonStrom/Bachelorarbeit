\section{Referenzmodellierung}\label{theorie:referenzmodellierung}
Gemäß des konstruktionsprozessorientierten Referenzmodellbegriffs von vom Brocke ist ein Referenzmodell als solches zu erkennen, wenn der Gegenstand und/oder\footnote{Die Verwendung von \enquote{und/oder} wurde hier gewählt, da der Autor der Quelle das \enquote{oder} aus der boolschen Algebra gewählt hat um explizit beide Fälle einzuschliessen.} der Inhalt des Referenzmodells bei der Konstruktion des Gegenstandes und/oder des Inhaltes eines zu konstruierenden Anwendungsmodells wiederverwendet werden kann.\footcite[Vgl.][34]{vomBrocke.2003} Dabei hat ein Referenzmodell einen Empfehlungscharakter und stellt eine \enquote{best practice} dar.\footcite[Vgl.][31]{vomBrocke.2003} 

Ein Referenzmodell kann nach vom Brocke nicht objektiv allgemeingültig sein und auch keinen objektiven Empfehlungscharakter haben, sondern muss subjektiv beurteilt werden.\footcite[Vgl. auch im Folgenden][31~f.]{vomBrocke.2003}  Dabei ist zumindest von den Interessensgruppen der Konstruierenden und der Nutzenden auszugehen, welche das Referenzmodell subjektiv unterschiedlich nach Allgemeingültigkeit und Empfehlungscharakter bewerten. Je nachdem welche Beurteilung höher gewichtet wird und früher einfließt, kann also entweder von der Situation ausgegangen werden, dass das Referenzmodell vom Konstruierenden zu einem solchen erklärt wird oder ein Modell, ob vom Konstruierenden beabsichtigt oder nicht, von den Nutzenden zu einem solchen erhoben wird.